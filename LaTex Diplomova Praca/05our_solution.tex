\chapter{Our approach}\label{chap:proposal}
The approach most often used to perform abductive reasoning in description logic without translating to other formal logics is based on the following thought:
Let there be a knowledge base $\K$, observation $\Obs$, and a set of axioms $\S$. If $\K \sqcup \S$ is consistent, then by definition there is at least one model of $\K \sqcup \S$.  If $\K \sqcup \S \sqcup \neg \Obs$ is inconsistent, then there is no model for $\K \sqcup \S$ where $\neg \Obs$ is not true, that is every model of $\K \sqcup \S$ contains $\Obs$, so $\K \sqcup \S \models \Obs$. \\ \\
We can find such a set $\S$ by generating every model of $\K \sqcup \neg \Obs$, and picking a set of complements of axiom in these models so that every model has at least one axiom complement in $\S$. This can be formulated as the hitting set problem (which is equivalent to the set cover problem) - for each model in the set of models $M$ of  $\K \sqcup \neg \Obs$, we create an \textbf{antimodel} consisting of negations of every axiom in the model, the set of these antimodels we call $M'$. Our goal is finding a minimal (inclusion-wise) set $\S$ containing at least one axiom from each antimodel in $M'$ - $\S$ a hitting set for $M$.
\\

Additionally, if $\S$ is relevant and explanatory (\ref{explanation}) and $\K \sqcup \S$ is consistent, it's an explanation for observation $\Obs$.
\\ 

\begin{mydef} \label{HS} (Hitting set  (TODO: cituj paper)) \\
A hitting set for a collection of
sets F (in our case, F is the collection of antimodels) is a set H s.t. $H \cap S \neq  \{\}$ for every S $\in$ F. A hitting set H for F is
minimal if there is no other hitting set H' for F s.t. $H' \not \subseteq H$.
\end{mydef}
\begin{mydef} \label{Antimodel} 
To better explain our algorithm, it will be useful to define the term 'antimodel'. Let M be a model of knowledge base $\K$. Then M' is an antimodel of $\K$ iff it contains negations of every axiom from M and nothing else - for every axiom of form $a:C$ in M, M' has $a:\neg C$, and for every axiom $a,b:R$ in M it has $a,b:\neg R$ (a and b are not in role R). (TODO: toto asi pre niektore deskripcne logiky nieje mozne vyjadrit, (role complements))
\end{mydef}
This idea was first introduced by Raymond Reiter in ''A Theory of Diagnosis from First Principles" \cite{Reiter1987} as a general method for abductive reasoning in any formal logic with binary semantics (every statement is either true or false) and operands $\land , \lor, \neg$  with their usual semantic meaning, including first order logic. Additionally, Reiter proposes using a hitting set tree, which we will describe later.


 Ken Halland and Katarina Britz in '' ABox abduction in ALC using a DL tableau" \cite{halland2012} proposed an algorithm using the idea of hitting sets for abduction in description logic ALC, using a modified tableaux algorithm - their algorithm first develops all possible completion graphs (multiple graphs resulting from the use of $\sqcup$ rule), based on the knowledge base, and once there are no more rules to apply, they add an axiom from the observation complement $\neg \Obs$ to the knowledge base. If there is no rule to apply or unused observation, the model is added to the list of models from which minimal hitting set is generated. \\ This algorithm generates every model reachable by tableaux algorithm for $\K \sqcup \Obs$, which as we will show, is not necessary. \\ \\
Our work is mainly based on the algorithm by Martin Homola and Júlia Pukancová, using the idea of a hitting set tree from Reiter \cite{Reiter1987}, which make generating every model of $\K \sqcup \neg \Obs$ not necessary. 

\begin{mydef} \label{HSTree} (Hitting set tree  (TODO: reference))
	A hitting set tree (HS-tree) for a collection of sets F is T = (V, E, L, H), where
	(V, E) is the smallest tree in which the labelling function L labels the nodes of V
	by elements of F, the edges of E by elements of sets in F, and H(n) is the set
	of edge-labels from the root node to n $\in$ V , s.t.: \begin{itemize} \item (a)
		 for the root r $\in$ V : L(r) = S for some S $\in$ F, if F $\neq \{\}$, otherwise L(r) = \{\}; 
\item(b) for each n $\in$ V : L(n) = S
	for some S $\in$ F s.t. S $\cap$ H(n) = \{\}, if such S $\in$ F exists, otherwise L(n) = \{\};
\item	(c) each n $\in$ V has a successor $n_\sigma $ for each $\sigma \in L(n)$ with $L(n, n_\sigma ) = \sigma$.
\end{itemize}
\end{mydef}

We can generate the HS-tree inf the following way: let F be the list of antimodels for $\K \cap \neg \Obs$ . The label of each vertex n is either an antimodel of $\K \cap H(n) \cap \neg \Obs$ (which is also an antimodel of $\K \cap \neg \Obs$) or empty, if no such model exists - in this case H(n) is a hitting set of F, and $\K \cap H(n) \models \Obs$ - H(n) is an explanation of H(n) if it also fulfills the other conditions (\ref{explanation}). If the vertex n is labeled by an antimodel, we create a child vertex for each axiom of antimodel L(n) - we label the edge by that axiom. \\
By using a HS-tree, we can find a hitting set that hits every antimodel of $\K \cap \neg \Obs$ while having to generate only a few of these models - each choice of edge label usually also eliminates many models that were not generated. 
The H and P (TODO ref) algorithm uses this algorithm, combined with the pruning methods (for example, eliminate a vertex n if H(n) contains axioms \{A , $\neg$ A \} (inconsistent with itself) or if there is a vertex n' with emply label (either due to pruning or being a hitting set) and $H(n') \subseteq H(n)$, or if there is a vertex n' where H(n) = H(n')).   
%\section{Explanation of our algorithm}


